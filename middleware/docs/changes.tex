The following changes will need to be made to \settings in order
to use this middleware. All changes are actually made in {\tt
  settings\_local.py}:
\begin{itemize}
\item\emph{Settings for middleware classes: } The
  IPAddressMiddleware resides in the file {\tt middleware.py},
  and can be added to the MIDDLEWARE\_CLASSES variable in
  \settings as 'middleware.IPAddressMiddleware'. The
  IPAddressMiddleWare can be added any where in the list of
  middleware classes. Thus, the MIDDLEWARE\_CLASSES variable could
  be set like:
  \begin{Verbatim}
    MIDDLEWARE_CLASSES = (
      'django.middleware.common.CommonMiddleware',
      'django.contrib.sessions.middleware.SessionMiddleware',
      'django.contrib.auth.middleware.AuthenticationMiddleware',
      'middleware.IPAddressMiddleware',
    )
  \end{Verbatim}
\item\emph{Specifying registered IPs: } This is given in the
  variable SRJ\_IP\_REG as a list, where each entry can be either a
  single IP address value specified as a string, or a sequence of
  two values as strings, which specify the start and end of an IP
  address range. Thus,
  \begin{Verbatim}
    SRJ_IP_REG = [
        '192.168.10.1', '192.168.10.4', ['192.168.10.6', '192.168.10.24'],
        '192.168.10.40'
        ]
  \end{Verbatim}
  Enhancements are also possible, e.g., the individual values can
  also themselves be ranges, e.g., '192.168.10.0/3'. For details,
  see the examples in
  \url{http://code.google.com/p/netaddr/wiki/IPv4Examples}. Still
  further extensions are possible, e.g., see
  \url{http://code.google.com/p/netaddr/wiki/WildcardExamples},
  and the complete API at
  \url{http://packages.python.org/netaddr/}. However, in the
  interests of simplicity, these are not currently included.
\item\emph{Templates for login pages: } There are separate login
  pages for users from registered, and unregistered IPs. These
  are configured through the variables SRJ\_TMPL\_LOGIN\_REG and
  SRJ\_TMPL\_LOGIN\_UNREG, e.g.,
  \begin{Verbatim}
    SRJ_TMPL_LOGIN_REG = 'login_reg.html'
    SRJ_TMPL_LOGIN_UNREG = 'login.html'
  \end{Verbatim}
  Please note that these are set in the middleware, and hence
  specifying templates as {\tt template\_name} arguments to the
  dictionary for the {\tt login} view will not work.
\end{itemize}
